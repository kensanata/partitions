\documentclass{notes}

\begin{document}
\title{Boismortier 2. Suite}
\maketitle
\tableofcontents
\section{Einf\"{u}hrung}
\section{Prélude}
Das Vorspiel der Suite.
\subsection{notes inégales}
In der franz"osischen Barockmusik werden Achtell"aufe nicht gleichm"a"sig gespielt. Wenn etwa in Takt 12
\begin{center}
\begin{lilypond}
\time 2/2
\clef "french"
\key g \major
\set Score.currentBarNumber = #12
\override Score.BarNumber.break-visibility = ##(#t #t #t)
\bar ""
\transpose c c'' {
	b,8^\markup { \teeny {\rhythm { 8[ 8] } als \rhythm { \tuplet 3/2 { 4 8 } } oder \rhythm { { 8. [16]} }}}
  d8 e fis g2 ~ | g8
  }
}
\override Score.BarNumber.break-visibility = ##(#f #f #f)
|
\end{lilypond}
\end{center}
gleichm"a"sig achtel notiert sind (das w"aren \textit {notes égales}), so wird eher mit einem \textit{swing feel} gespielt.
\begin{center}
\begin{lilypond}
\time 2/2
\clef "french"
\key g \major
\set Score.currentBarNumber = #7
\override Score.BarNumber.break-visibility = ##(#t #t #t)
\bar ""
\transpose c c'' {
	b,8^\markup {
		\teeny {
			\rhythm { 8 [8] }
			als \rhythm { \tuplet 3/2 { 4 8 } }
			oder \rhythm {  8. [16]}}}
	d8 e fis g2  |
  \set Score.currentBarNumber = #7
  b,8. [d16] e8. [fis16] g2 |
  \set Score.currentBarNumber = #7
  \times 2/3 {b,4 d8} \times 2/3 {e4 fis8} g2
  }
\override Score.BarNumber.break-visibility = ##(#f #f #f)
\end{lilypond}
\end{center}
Die exakte Ausf"uhrung ist historisch nicht immer einwandfrei "uberliefert,
sodass der Interpret interpretieren muss. In schnellen Passagen verliert sich
die ungleiche Ausf"uhrung auch wieder.
Auf der Blockfl"ote kann mit den Silben \textit{t"u - r"u} angeblasen werden.
Wobei das \textit{t"u} l"anger und das \textit{r"u} langsamer ausgef"uhrt wird.
Zuerst wird auf einem Ton ge"ubt bevor L"aufe ge"ubt werden.

\begin{center}
  \begin{lilypond}
    <<
    \new Voice=up {
      \key g \major
      \clef "french"
      \cadenzaOn
      \override Score.BarNumber.break-visibility = ##(#t #t #t)
      \bar ""
      \transpose c c'' {
        g,8^\markup {
		\teeny {
			\rhythm { 8 [8] }
			als \rhythm { \tuplet 3/2 { 4 8 } }
			}}
			[g, g, g,] g, [g, g, g,] \bar "||"
        g,8 [a, b, c] d [e fis g]
      }
    }\addlyrics {tü rü tü rü tü rü tü rü tü rü tü rü tü rü tü rü}
    >>
  \end{lilypond}
\end{center}
\subsection{tierces coulée}
Eine \textit{tierce coulée\footnote{Als \enquote{Coulez} bezeichnet Hotteterre Noten die ohne extra Zungenschlag ausgef"uhrt werden.}} (in etwa: \enquote{rinnende/kullernde Terz}) ist
eine Verzierung (Ornament) eines Terzfalls gleicher Notenwerte, bei dem ein
Sekundenschritt gespielt wird, etwa in Takt 3 der Pr\'{e}lude
\begin{center}
\begin{lilypond}
\time 2/2
\key g \major
\clef "french"
\set Score.currentBarNumber = #3
\override Score.BarNumber.break-visibility = ##(#t #t #t)
\bar ""
\transpose c c'' {b,4 \appoggiatura a,8 g,4 b,4. c8}
\override Score.BarNumber.break-visibility = ##(#f #f #f)
|
\end{lilypond}
\end{center}
\section{Rondeau}
\section{Passecailles}
Dieses St"uck ist in den historischen Noten auf IMSLP mit einem Vorzeichen notiert.
\section{Bourrée}
\section{Brunette}
Eine \textit{brunette} ist eine Liedform die sich im 17. / 18. Jahrhundert in
Frankreich gewisser Beliebtheit erfreute.
\end{document}
